\documentclass{article}
\usepackage[utf8]{inputenc}
\usepackage{natbib}
\usepackage{graphicx}
\usepackage{lastpage}
\usepackage{xcolor}
\usepackage{lipsum}
\usepackage[T1]{fontenc}
\usepackage{fancyhdr}
\usepackage{geometry}
\usepackage{url}
\usepackage[dutch]{babel}


\usepackage{color}
\usepackage{pdfpages} %om pdf files met requirements lijsten toe te voegen




\bibliographystyle{abbrv} %abbrvnat geeft problemen

\title{Software Requirements Specification}
\author{} %leave empty
\date{19 november 2014} %ok, manuele datum

\addtolength{\footskip}{1.3cm} % make more space for the footer
\pagestyle{fancyplain} % use fancy for all pages except chapter start
\lhead{}
\cfoot{\includegraphics[height=1.3cm]{Small_Logo.png}} % right logo
\rfoot{\thepage}
\renewcommand{\headrulewidth}{0.3pt} % remove rule below header
\renewcommand{\footrulewidth}{0.3pt} % remove rule below header

\begin{document}

\makeatletter
\begin{titlepage}

\newcommand{\HRule}{\rule{\linewidth}{0.7mm}} % Defines a new command for the horizontal lines, change thickness here


\vspace*{1.2mm}

\center 
\includegraphics[scale=0.6]{Logo.png}\\[1cm] 
%---------------------------------------------------------------------------------------
%	HEADINGS SECTION
%----------------------------------------------------------------------------------------

\textsc{\LARGE Vrije Universiteit Brussel}\\[0.3cm] % Name of your university/college
\textsc{\large WE-DINF-6537}\\
\textsc{\large Project Software Engineering}\\
\textsc{\large Academiejaar 2014-2015}\\[0.3cm] 
%\textsc{\large Software Engineering}\\[0.7cm] % Major heading such as course name

%----------------------------------------------------------------------------------------
%	TITLE SECTION
%----------------------------------------------------------------------------------------

\HRule \\[0.4cm]
{ \huge \bfseries \@title \\[0.5cm] }
\HRule \\[0.5cm]
 
%----------------------------------------------------------------------------------------
%	AUTHOR SECTION
%----------------------------------------------------------------------------------------

\Large
% laat voorlopig nog even de namen/mailadressen op deze pagina staan
% eerst moeten we zien of er een beter alternatief is om de lege plek dan op te vullen
% anders laten we ze gewoon staan...

%volgens mij ziet het er zo heel goed uit, maar als ze echt wegmoeten mss vub logo groter maken? 
% => OK, ik vind het ook  beter zo, we laten ze staan!
Douglas Horemans \textit{<dhoreman@vub.ac.be>}\\
Hannah Pinson \textit{<hpinson@vub.ac.be>}\\
Ivo Vervlimmeren \textit{<ivervlim@vub.ac.be>}\\
Noah Van Es \textit{<noahves@vub.ac.be>}\\
Pieter Steyaert \textit{<psteyaer@vub.ac.be>}\\

\vspace{0.6cm}

\includegraphics[scale=0.4]{VUB_schild.pdf}\\[0.5cm]

{\large 19 november 2014}
\vfill % Fill the rest of the page with whitespace

\end{titlepage}

\newpage
\section*{Versiegeschiedenis}
\addcontentsline{toc}{section}{Versiegeschiedenis}

\begin{center}
\begin{tabular}[t]{| c | c | c | c |}
    \hline
    \textbf{Versie} & \textbf{Datum} & \textbf{Auteurs\cite{note:author}} & \textbf{Beschrijving} \\
    \hline
    
    1.0     &  05/11/2014   &   \begin{tabular}{c} 
                                    Hannah Pinson \\
                                    Ivo Vervlimmeren \\
                                    Noah Van Es \\ 
                                    Pieter Steyaert \\
                                \end{tabular} & Aanmaak eerste versie \\
    \hline
    1.1     &  18/11/2014   &   \begin{tabular}{c} 
                                    Hannah Pinson \\
                                     \end{tabular} & \begin{tabular}{c}  
                                     			Gekregen feed-back  gedeeltelijk doorgevoerd\\
                                     			planning eerste sprint aangevuld \\
							 \end{tabular}  \\
    \hline

\end{tabular}
\end{center}
\newpage

\tableofcontents
\newpage

%%%%%%%%%%%%%%%%%%%%%%%%%%%%%%%%%% INTRODUCTIE

\section{Introductie}

%%%
\subsection{Doel en doelpubliek} %van het SRS, niet van het project
Dit is het Software Requirements Specification (SRS) document voor SKRIBL,  een softwareproject uitgevoerd door de groep SE4\_1415 in het kader van het opleidingsonderdeel Software Engineering van de Vrije Universiteit Brussel. Dit document is opgesteld volgens de IEEE 1016-2009 standaard. \newline
\\
\noindent De algemene eisen die de klant aan het op te leveren product stelt zijn te vinden in de projectomschrijving van het opleidingsonderdeel \cite{Xtreport:organisatie}. Dit SRS biedt een globaal, gestructureerd overzicht van deze vereisten (user requirements) in onderdeel (\ref{subsec:userRequirements}). Daarnaast is er een gedetailleerde oplijsting van system requirements die voortvloeien uit deze user requirements, en die dienen als richtlijnen bij het ontwikkelen en testen van het softwareproduct, in onderdeel (\ref{sec:systemRequirements}). In overeenstemming met de principes van een agile development proces worden de system requirements in dit SRS \emph{bij iedere sprint} aangevuld. Meer informatie over het doel en de planning van deze sprints is te vinden in het Software Project Management Plan \cite{Xtreport:SPMP}. \newline
\\
\noindent Dit document is zowel bedoeld voor de klant en externe controle als voor de interne organisatie. In het bijzonder worden de system requirements in onderdeel (\ref{sec:systemRequirements}) door alle leden van het team gebruikt als richtlijnen bij het volledige plannings-, ontwikkelings- en testproces van iedere sprint. De globale oplijsting van requirements in onderdeel (\ref{subsec:userRequirements}) dient als handvest voor de grote lijnen van het software design en de algemene planning van het project.  

%%%
\subsection{Product Scope}
Het doel van dit softwareproject is het ontwikkelen van SKRIBL, een webapplicatie die het enerzijds mogelijk maakt voor onderzoekers om wetenschappelijke publicaties te beheren en die anderzijds de netwerken van deze onderzoekers analyseert en op een aantrekkelijke manier visualiseert. Deze applicatie wordt binnen het kader van het opleidingsonderdeel software engineering gecre\"{e}erd gedurende het academiejaar 2014-2015.

%%%
\subsection{Gebruikte conventies en afkortingen}
Suggesties en opmerkingen voor toekomstige aanpassingen in dit document worden aangeduid met vierkante haakjes en een cursief lettertype: [{\it voorbeeld suggestie}]. \newline
\\
\noindent Volgende afkortingen worden in dit SRS gebruikt:
\begin{itemize}
\item SRS: Software Requirements Specification (document)
\item SPMP: Software Project Management Plan (document)
%\item STD:  Software Test Plan (document)
%\item SDD: Software Design Document (document)
\item FR-U: Functional Requirement, type User
\item FR-P: Functional Requirement, type Publicatie
\item FR-D: Functional Requirement, type Data-Mining
\item NFR-S: Niet-Functionele Requirement, type Security
\item NFR-R: Niet-Functionele Requirement, type Reliability
\item NFR-P: Niet-Functionele Requirement, type Performance
\item G: aangemelde gebruiker
\item B: bezoeker, niet-geauthenticeerde persoon 
\end{itemize}


%%%
\subsection{Referenties}
\begingroup
\renewcommand{\section}[2]{}% verwijdert titel sectie referenties
\bibliography{referenties}
\endgroup

\clearpage
%%%%%%%%%%%%%%%%%%%%%%%%%%%%%%%%%% ALGEMENE BESCHRIJVING 

\section{Algemene Beschrijving}

\subsection{Perspectief van het product}
De ontwikkelde webapplicatie is een op zichzelf staand softwareproduct. Het maakt geen deel uit van andere softwareproducten maar steunt voor een deel van zijn content (i.e., aangeleverde publicaties) op het contacteren van andere websites met gelijkaardige functionaliteiten. \newline
\\
%----comment----%
\noindent [{\it Volgens de IEEE standaard moeten in de onderdeel verder nog de interfaces (System interfaces; User interfaces; Hardware interfaces; Software interfaces;Communications interfaces) beschreven worden. Indien van toepassing zullen deze onderdelen, in samenspraak met de Software Architect en Configuration Manager, in latere versies aangevuld worden.}]

\subsection{Functies van het product}
\label{subsec:userRequirements}

Hieronder volgt een gestructureerde oplijsting van de functionele vereisten zoals beschreven in de projectomschrijving van het opleidingsonderdeel \cite{Xtreport:organisatie}.

\renewcommand{\labelitemi}{$\diamond$}
\renewcommand{\labelitemii}{$\bullet$}
\renewcommand{\labelitemiii}{$\cdot$}

\begin{itemize}
    %%
    %USER %
    %%
    \item USER
    \begin{itemize}
        \item inloggen en uitloggen 
        \item account aanmaken
        \item account beheren 
        \item aanleggen en beheren van portfolio eigen publicaties
        \item persoonlijke score via portfolio
	\item toevoegen en beheren van lijst publicaties van derden ("favorieten")
	\item publicaties opslaan op eigen computer, buiten applicatie
	\item top drie relevantste publicaties binnen onderzoeksdomein 
	\item suggesties van relevante papers en feed-back/voorkeuren 
	\item opzoeken en toevoegen van publicaties, gevonden in systeem en/of internet, via invulformulier of reeds toegevoegde publicaties
	\item annoteren van publicaties, toevoegen van bijlagen 
	\item publicaties linken op manieren die systeem niet standaard voorziet
	\item raadplegen, genereren en exporteren (PDF) van persoonlijk statistieken en bijhorende grafieken
	\item visualisatie van en interactie met sociaal netwerk in een graaf
	\item mobiele interface
    \end{itemize}
   %%
    %PUBLICATIES %
    %%
    \item PUBLICATIES 
    \begin{itemize}
    \item toevoegen van content + metadata door extractie uit PDF/BibTex en/of manuele aanvulling
    \item weergave van publicaties 
    \item downloaden van het internet
    \item opslaan op computer vd gebruiker (buiten applicatie)
    \end{itemize}
     %%
    %DATAMINING %
    %%
    \item DATAMINING 
    \begin{itemize}
    \item persoonlijke score van gebruiker berekenen adhv portfolio: 
    	\begin{itemize}
    		\item aantal eigen publicaties gedeeld door het aantal maanden sinds de eerste publicatie
   		 \item kwaliteit op basis van de classificatie van conferences en journals
    		\item impact van de eigen publicatie (aantal citaties)
    	\end{itemize}
    \item relevantie van publicatie voor gebruiker berekenen in functie van onderzoeksdomein 
    \item relevantie van publicatie voor gebruiker berekenen op basis van co-auteurs, keywords, ... en dynamische voorkeuren gebruiker 
    \item statistieken voor gebruiker berekenen adhv volgende metrieken (zie ook persoonlijke score)
      	\begin{itemize}
    		\item publicaties per jaar
   		 \item ranking van de publicaties (afhankelijk van de ranking van conference/journal)
    		\item aantal citaties
    	\end{itemize}
    \end{itemize}
\end{itemize}




\subsection{Gebruikers}
De beoogde gebruikers zijn onderzoekers actief in de wetenschappelijke wereld. Deze vormen de enige klasse van gerechtmatigde gebruikers. Daarnaast worden er verschillende veiligheidsmaatregelen ingebouwd om niet-rechtmatige gebruikers de toegang tot de applicatie te ontzeggen. 

\subsection{Omgeving}
Aan de back-end draait het systeem op Wilma, een server beschikbaar gesteld aan de studenten wetenschappen van de Vrije Universiteit Brussel.  Front-end ondersteunt de applicatie alle gangbare en up-to-date browsers. De mobiele interface wordt ontwikkeld voor Android smartphones.

\subsection{Beperkingen op design en implementatie}
JavaScript, HTML5, CSS en bijbehorende open-source frameworks en bibliotheken zijn de enige programmeertalen en technologie\"{e}n die gebruikt mogen worden. In het algemeen mag enkel vrije software aangewend worden, en deze software moet ook verantwoord kunnen worden. Er moet daarnaast ten allen tijde gebruik worden gemaakt van een testing framework. Code moet volgens een vooraf vastgelegde standaard voorzien worden van commentaar. Ten slotte moet GitHub gebruikt worden als (publieke) repository. 

\subsection{Gebruikshandleidingen}

[{\it nog te bepalen}]


\clearpage


%%%%%%%%%%%%%%%%%%%%%%%%%%%%%%%%%% SYSTEEM REQUIREMENTS
\section{Specifieke Requirements}
\label{sec:systemRequirements}

\noindent De functionele requirements zijn opgedeeld in drie types: gebruikers (FR-U), publicaties (FR-P) en data-mining (FR-D); de niet-functionele requirements zijn opgedeeld in de types reliability (NFR-R), performance (NFR-P) en security (NFR-S).  \\
\\
\noindent De hierop volgende secties beschrijven aan de hand van use cases de functionele requirements in detail. Hierbij zijn de prioriteiten niet vermeld, omdat aan sommige onderdelen of varianten van de beschreven scenario's andere prioriteiten toegekend werden. Op het bijgevoegde requirements dashboard kunnen de specifieke prioriteiten afgelezen worden. De prioriteiten worden aangegeven met kleuren en bijhorende lettters:  donkerblauw + H  = hoge prioriteit;  lichtblauw + M = gemiddelde prioriteit;  wit + L = lage prioriteit.


\subsection{Functionele Requirements: USER}
\vspace{4 mm}


%%%%%%%%%%%% REGISTRATIE
\subsubsection{REGISTRATIE}
\vspace{2 mm}

\textbf{ID}: FR-U001
\vspace{2 mm}

\noindent \textbf{Betrokkenen}: bezoeker (B); systeem (S) 
\vspace{2 mm}

\noindent \textbf{Samenvatting}: B voert gewenste account-informatie in. Deze informatie wordt gevalideerd, en NG kan invoer aanpassen waar nodig. Indien alle informatie aan vereisten voldoet wordt een nieuw account geregistreerd. 
\vspace{2 mm}

\noindent \textbf{Pre-conditie}: B bevindt zich op de eerste pagina van de Skribl applicatie. 
\vspace{2 mm}

\noindent \textbf{Trigger}: B geeft aan zich te willen registreren (button)
\vspace{4 mm}

\hrule
\vspace{2 mm}
\noindent \textbf{Scenario}:
\begin{enumerate}
\item B krijgt de mogelijkheid om volgende (verplichte) velden aan te vullen:
\begin{description}
  \item[voornaam en achternaam]  \hfill \\
  {\it moet voldoen aan vereisten voor algemene namen*}
  \item[onderzoeksgroep, departement, faculteit en instelling] \hfill \\
   {\it moet voldoen aan vereisten voor algemene namen*}
  \item[algemene en specifieke onderzoeksdomeinen] \hfill \\
  {\it moet voorkomen in een lijst met algemene en bijhorende specifieke onderzoeksdomeinen}
  \item[username] \hfill \\
    {\it moet uniek zijn; enkel letters, cijfers en underscores toegelaten}
    \item[email] \hfill \\
  {\it moet voldoen aan de standaardvorm voor een e-mailadres.}
       \item[taalvoorkeur] \hfill \\
   {\it NL of EN}
       \item[wachtwoord] \hfill \\
  {\it moet minstens 1 cijfer bevatten; minimum 6 en maximum 20 karakters lang}\\
\end{description}
  \noindent {\it Er mogen geen velden opengelaten worden. *een algemene naam kan accenten, koppeltekens en taal-gebonden speciale karakters bevatten, maar geen cijfers of andere speciale karakters.} 
  
 \item B kan deze informatie doorvoeren (button)  $\hookrightarrow$ A
 \item S controleert en registreert nieuw account  $\hookrightarrow$ B
 \item B krijgt melding van succesvolle registratie 
\end{enumerate}
\noindent \textbf{Alternatief verloop}:
\begin{description}
\item[ $\hookrightarrow$ A] De ingevoerde informatie (uniciteit username wordt niet beschouwd) voldoet niet aan de vereisten. B krijgt een melding en kan informatie aanpassen. $\hookrightarrow$ 1.
\item[ $\hookrightarrow$ B] De ingevoerde username is niet uniek. B krijgt een melding en kan username aanpassen. $\hookrightarrow$ 1.
\end{description}

\vspace{2 mm}
\hrule
\vspace{4 mm}

\noindent \textbf{Post-conditie}: B kan inloggen met eigen username en wachtwoord. \\




%%%%%%%%%%%% LOGIN
\subsubsection{LOGIN}
\vspace{2 mm}

\textbf{ID}: FR-U002
\vspace{2 mm}

\noindent \textbf{Betrokkenen}: bezoeker (B); systeem (S) 
\vspace{2 mm}

\noindent \textbf{Samenvatting}: B vraagt toegang tot de applicatie door invoer van username en wachtwoord. Enkel na invoer van een correcte combinatie wordt de toegang verleend. 
\vspace{2 mm}

\noindent \textbf{Pre-conditie}: B bevindt zich op de eerste pagina van de Skribl applicatie. 
\vspace{2 mm}

\noindent \textbf{Trigger}: B geeft aan zich te willen inloggen (button)

\vspace{4 mm}
\hrule
\vspace{2 mm}
\noindent \textbf{Scenario}:
\begin{enumerate}
\item B krijgt de mogelijkheid om username en wachtwoord in te vullen    
 \item B kan deze informatie doorvoeren (button)
  \item S voert authenticatie uit en verleent toegang voor correcte combinatie username en wachtwoord   $\hookrightarrow$ A
\end{enumerate}
\noindent \textbf{Alternatief verloop}:
\begin{description}
\item[ $\hookrightarrow$ A] De combinatie username / wachtwoord is niet correct. B krijgt een melding en kan informatie aanpassen. $\hookrightarrow$ 1.
\end{description}

\vspace{2 mm}
\hrule
\vspace{4 mm}

\noindent \textbf{Post-conditie}: B is ingelogd en komt op persoonlijk dashboard terecht. \\



%%%%%%%%%%%% LOGOUT
\subsubsection{LOGOUT}
\vspace{2 mm}

\textbf{ID}: FR-U003
\vspace{2 mm}

\noindent \textbf{Betrokkenen}: aangemelde gebruiker (G); systeem (S) 
\vspace{2 mm}

\noindent \textbf{Samenvatting}: G kan op elk moment afmelden.  
\vspace{2 mm}

\noindent \textbf{Pre-conditie}: G is ingelogd. 
\vspace{2 mm}

\noindent \textbf{Trigger}: G geeft aan zich te willen uitloggen (button)

\vspace{4 mm}
\hrule
\vspace{2 mm}
\noindent \textbf{Scenario}:
\begin{enumerate}
  \item S sluit sessie af. 
\end{enumerate}


\vspace{2 mm}
\hrule
\vspace{4 mm}

\noindent \textbf{Post-conditie}: G bevindt zich op de eerste pagina van de Skribl applicatie en is niet meer aangemeld. \\



%%%%%% ACCOUNT VERWIJDEREN 
\subsubsection{ACCOUNT VERWIJDEREN}
\vspace{2 mm}

\textbf{ID}: FR-U004
\vspace{2 mm}

\noindent \textbf{Betrokkenen}: aangemelde gebruiker (G); systeem (S) 
\vspace{2 mm}

\noindent \textbf{Samenvatting}: G kan account verwijderen.  
\vspace{2 mm}

\noindent \textbf{Pre-conditie}: G bevindt zich op dashboard.
\vspace{2 mm}

\noindent \textbf{Trigger}: G geeft aan account te willen verwijderen (button)

\vspace{4 mm}
\hrule
\vspace{2 mm}
\noindent \textbf{Scenario}:
\begin{enumerate}
  \item G krijgt vraag om beslissing te bevestigen 
  \item S verwijdert account 
  \item G krijgt melding van succesvolle verwijdering account
\end{enumerate}


\vspace{2 mm}
\hrule
\vspace{4 mm}

\noindent \textbf{Post-conditie}: G bevindt zich op de eerste pagina van de Skribl applicatie, is niet meer ingelogd en kan voorgaande username en wachtwoord niet meer gebruiken om in te loggen. \\


%%%%%% ACCOUNTGEGEVENS WIJZIGEN

\subsubsection{ACCOUNTGEGEVENS WIJZIGEN}
\vspace{2 mm}

\textbf{ID}: FR-U005
\vspace{2 mm}

\noindent \textbf{Betrokkenen}: aangemelde gebruiker (G); systeem (S) 
\vspace{2 mm}

\noindent \textbf{Samenvatting}: G kan de gegevens van zijn/haar account aanpassen. 
\vspace{2 mm}

\noindent \textbf{Pre-conditie}: G bevindt zich op persoonlijk dashboard.
\vspace{2 mm}

\noindent \textbf{Trigger}: G geeft aan accountgegevens te willen wijzigen (button)

\vspace{4 mm}
\hrule
\vspace{2 mm}
\noindent \textbf{Scenario}:
\noindent Dit scenario is analoog aan het scenario REGISTRATIE. De velden zijn in dit geval ingevuld met reeds aanwezige accountgegevens; de username van een gebruiker kan niet gewijzigd worden. 

\vspace{2 mm}
\hrule
\vspace{4 mm}

\noindent \textbf{Post-conditie}: G krijgt een melding van succesvolle aanpassing. \\



%%%%%% TAALKEUZE

\subsubsection{TAALKEUZE}
\vspace{2 mm}

\textbf{ID}: FR-U006
\vspace{2 mm}

\noindent \textbf{Betrokkenen}: bezoeker (B) of aangemelde gebruiker (G); systeem (S) 
\vspace{2 mm}

\noindent \textbf{Samenvatting}: De interface kan in twee talen, Engels en Nederlands, aangeboden worden. 
\vspace{2 mm}

\noindent \textbf{Pre-conditie}: Indien G, dan wordt de interface aangeboden volgens de taalvoorkeur van de gebruiker, meegegeven bij registratie.
\vspace{2 mm}

\noindent \textbf{Trigger}: G/B geeft aan de taal te willen wijzigen naar Engels of Nederlands (button)

\vspace{4 mm}
\hrule
\vspace{2 mm}
\noindent \textbf{Scenario}:
\noindent S laadt een nieuwe interface waarin de taal aangepast is.
\vspace{2 mm}
\hrule
\vspace{4 mm}

\noindent \textbf{Post-conditie}: G/B ziet een interface in de door hem/haar gekozen taal. \\






\clearpage


%%%%%%%%%%%%%%%%PUBLICATIE%%%%%%%%%%%%%%%%%%%%


\subsection{Functionele Requirements: PUBLICATIE}
\vspace{4 mm}

%%%%%%%%%%%% PUBLICATIE UPLOADEN 
\subsubsection{PUBLICATIE UPLOADEN}
\vspace{2 mm}

\textbf{ID}: FR-P001
\vspace{2 mm}

\noindent \textbf{Betrokkenen}: aangemelde gebruiker (G); systeem (S) 
\vspace{2 mm}

\noindent \textbf{Samenvatting}: G kan een bestand uploaden in PDF- of Bibtex-formaat, waarna S metadata uit dit bestand extraheert; de gevonden metadata kan manueel aangevuld worden door de gebruiker. Het bestand en metadata worden opgeslagen door S. 
\vspace{2 mm}

\noindent \textbf{Pre-conditie}: G bevindt zich op persoonlijk dashboard.  
\vspace{2 mm}

\noindent \textbf{Trigger}: G geeft aan een publicatie te willen uploaden (button)
\vspace{4 mm}

\hrule
\vspace{2 mm}
\noindent \textbf{Scenario}:
\begin{enumerate}
\item G krijgt de mogelijkheid om bestand uit eigen filesystem up te loaden.
\item S controleert of bestand PDF of Bibtex formaat heeft $\hookrightarrow$ A
\item S extraheert metadata, G kan metadata verder aanvullen:
\begin{description}
  \item[titel (V)]  \hfill \\
  {\it verplicht, moet voldoen aan vereisten voor algemene namen*}
  \item[auteurs (V) ] \hfill \\
   {\it verplicht, moet voldoen aan vereisten voor algemene namen*}
     \item[onderwerpen / keywords] \hfill \\
   {\it moet voldoen aan vereisten voor algemene namen*}
  \item[onderzoeksdomein] \hfill \\
  {\it moet voldoen aan vereisten voor algemene namen*}
   \item[titel van journal of proceedings (V) ] \hfill \\
   {\it verplicht, moet voldoen aan vereisten voor algemene namen*}
      \item[jaar van publicatie ] \hfill \\
   {\it moet een jaartal zijn}
      \item[URL] \hfill \\
   {\it moet voldoen aan de standaardvorm van een URL}
  \end{description}
  \noindent {\it *een algemene naam kan accenten, koppeltekens en taal-gebonden speciale karakters bevatten, maar geen cijfers of andere speciale karakters.} \\
  \\
  \noindent  [{\it De specificaties over het verzamelen van metadata zijn niet definitief. Deze zullen gewijzigd en aangevuld worden eens beslist is welke metadata een rol zullen spelen in het uitbouwen van het netwerk. }]
 \item G kan deze informatie doorvoeren (button)  $\hookrightarrow$ B
 \item S slaat publicatie op   
 \item B krijgt melding van succesvol toevoegen publicatie 
\end{enumerate}
\noindent \textbf{Alternatief verloop}:
\begin{description}
\item[ $\hookrightarrow$ A] Het bestand heeft niet het juiste formaat. G krijgt een melding. $\hookrightarrow$ 1.
\item[ $\hookrightarrow$ B] De ingevoerde metadata voldoet niet aan de vereisten. G krijgt een melding en kan aanpassingen doorvoeren. $\hookrightarrow$ 3.
\end{description}

\vspace{2 mm}
\hrule
\vspace{4 mm}


\noindent \textbf{Post-conditie}: Een nieuw bestand met bijhorende metadata is opgeslagen in het systeem. \\

\clearpage







\includepdf[pages={-}]{requirements_dashboard_15dec.pdf}








\end{document}
