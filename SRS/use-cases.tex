
\subsection{Functionele Requirements: USER}
\vspace{4 mm}


%%%%%%%%%%%% REGISTRATIE
\subsubsection{REGISTRATIE}
\vspace{2 mm}

\textbf{ID}: FR-U001
\vspace{2 mm}

\noindent \textbf{Betrokkenen}: bezoeker (B); systeem (S) 
\vspace{2 mm}

\noindent \textbf{Samenvatting}: B voert gewenste account-informatie in. Deze informatie wordt gevalideerd, en NG kan invoer aanpassen waar nodig. Indien alle informatie aan vereisten voldoet wordt een nieuw account geregistreerd. 
\vspace{2 mm}

\noindent \textbf{Pre-conditie}: B bevindt zich op de eerste pagina van de Skribl applicatie. 
\vspace{2 mm}

\noindent \textbf{Trigger}: B geeft aan zich te willen registreren (button)
\vspace{4 mm}

\hrule
\vspace{2 mm}
\noindent \textbf{Scenario}:
\begin{enumerate}
\item B krijgt de mogelijkheid om volgende (verplichte) velden aan te vullen:
\begin{description}
  \item[voornaam en achternaam]  \hfill \\
  {\it moet voldoen aan vereisten voor algemene namen*}
  \item[onderzoeksgroep, departement, faculteit en instelling] \hfill \\
   {\it moet voldoen aan vereisten voor algemene namen*}
  \item[algemene en specifieke onderzoeksdomeinen] \hfill \\
  {\it moet voorkomen in een lijst met algemene en bijhorende specifieke onderzoeksdomeinen}
  \item[username] \hfill \\
    {\it moet uniek zijn; enkel letters, cijfers en underscores toegelaten}
    \item[email] \hfill \\
  {\it moet voldoen aan de standaardvorm voor een e-mailadres.}
       \item[taalvoorkeur] \hfill \\
   {\it NL of EN}
       \item[wachtwoord] \hfill \\
  {\it moet minstens 1 cijfer bevatten; minimum 6 en maximum 20 karakters lang}\\
\end{description}
  \noindent {\it Er mogen geen velden opengelaten worden. *een algemene naam kan accenten, koppeltekens en taal-gebonden speciale karakters bevatten, maar geen cijfers of andere speciale karakters.} 
  
 \item B kan deze informatie doorvoeren (button)  $\hookrightarrow$ A
 \item S controleert en registreert nieuw account  $\hookrightarrow$ B
 \item B krijgt melding van succesvolle registratie 
\end{enumerate}
\noindent \textbf{Alternatief verloop}:
\begin{description}
\item[ $\hookrightarrow$ A] De ingevoerde informatie (uniciteit username wordt niet beschouwd) voldoet niet aan de vereisten. B krijgt een melding en kan informatie aanpassen. $\hookrightarrow$ 1.
\item[ $\hookrightarrow$ B] De ingevoerde username is niet uniek. B krijgt een melding en kan username aanpassen. $\hookrightarrow$ 1.
\end{description}

\vspace{2 mm}
\hrule
\vspace{4 mm}

\noindent \textbf{Post-conditie}: B kan inloggen met eigen username en wachtwoord. \\




%%%%%%%%%%%% LOGIN
\subsubsection{LOGIN}
\vspace{2 mm}

\textbf{ID}: FR-U002
\vspace{2 mm}

\noindent \textbf{Betrokkenen}: bezoeker (B); systeem (S) 
\vspace{2 mm}

\noindent \textbf{Samenvatting}: B vraagt toegang tot de applicatie door invoer van username en wachtwoord. Enkel na invoer van een correcte combinatie wordt de toegang verleend. 
\vspace{2 mm}

\noindent \textbf{Pre-conditie}: B bevindt zich op de eerste pagina van de Skribl applicatie. 
\vspace{2 mm}

\noindent \textbf{Trigger}: B geeft aan zich te willen inloggen (button)

\vspace{4 mm}
\hrule
\vspace{2 mm}
\noindent \textbf{Scenario}:
\begin{enumerate}
\item B krijgt de mogelijkheid om username en wachtwoord in te vullen    
 \item B kan deze informatie doorvoeren (button)
  \item S voert authenticatie uit en verleent toegang voor correcte combinatie username en wachtwoord   $\hookrightarrow$ A
\end{enumerate}
\noindent \textbf{Alternatief verloop}:
\begin{description}
\item[ $\hookrightarrow$ A] De combinatie username / wachtwoord is niet correct. B krijgt een melding en kan informatie aanpassen. $\hookrightarrow$ 1.
\end{description}

\vspace{2 mm}
\hrule
\vspace{4 mm}

\noindent \textbf{Post-conditie}: B is ingelogd en komt op persoonlijk dashboard terecht. \\



%%%%%%%%%%%% LOGOUT
\subsubsection{LOGOUT}
\vspace{2 mm}

\textbf{ID}: FR-U003
\vspace{2 mm}

\noindent \textbf{Betrokkenen}: aangemelde gebruiker (G); systeem (S) 
\vspace{2 mm}

\noindent \textbf{Samenvatting}: G kan op elk moment afmelden.  
\vspace{2 mm}

\noindent \textbf{Pre-conditie}: G is ingelogd. 
\vspace{2 mm}

\noindent \textbf{Trigger}: G geeft aan zich te willen uitloggen (button)

\vspace{4 mm}
\hrule
\vspace{2 mm}
\noindent \textbf{Scenario}:
\begin{enumerate}
  \item S sluit sessie af. 
\end{enumerate}


\vspace{2 mm}
\hrule
\vspace{4 mm}

\noindent \textbf{Post-conditie}: G bevindt zich op de eerste pagina van de Skribl applicatie en is niet meer aangemeld. \\



%%%%%% ACCOUNT VERWIJDEREN 
\subsubsection{ACCOUNT VERWIJDEREN}
\vspace{2 mm}

\textbf{ID}: FR-U004
\vspace{2 mm}

\noindent \textbf{Betrokkenen}: aangemelde gebruiker (G); systeem (S) 
\vspace{2 mm}

\noindent \textbf{Samenvatting}: G kan account verwijderen.  
\vspace{2 mm}

\noindent \textbf{Pre-conditie}: G bevindt zich op dashboard.
\vspace{2 mm}

\noindent \textbf{Trigger}: G geeft aan account te willen verwijderen (button)

\vspace{4 mm}
\hrule
\vspace{2 mm}
\noindent \textbf{Scenario}:
\begin{enumerate}
  \item G krijgt vraag om beslissing te bevestigen 
  \item S verwijdert account 
  \item G krijgt melding van succesvolle verwijdering account
\end{enumerate}


\vspace{2 mm}
\hrule
\vspace{4 mm}

\noindent \textbf{Post-conditie}: G bevindt zich op de eerste pagina van de Skribl applicatie, is niet meer ingelogd en kan voorgaande username en wachtwoord niet meer gebruiken om in te loggen. \\


%%%%%% ACCOUNTGEGEVENS WIJZIGEN

\subsubsection{ACCOUNTGEGEVENS WIJZIGEN}
\vspace{2 mm}

\textbf{ID}: FR-U005
\vspace{2 mm}

\noindent \textbf{Betrokkenen}: aangemelde gebruiker (G); systeem (S) 
\vspace{2 mm}

\noindent \textbf{Samenvatting}: G kan de gegevens van zijn/haar account aanpassen. 
\vspace{2 mm}

\noindent \textbf{Pre-conditie}: G bevindt zich op persoonlijk dashboard.
\vspace{2 mm}

\noindent \textbf{Trigger}: G geeft aan accountgegevens te willen wijzigen (button)

\vspace{4 mm}
\hrule
\vspace{2 mm}
\noindent \textbf{Scenario}:
\noindent Dit scenario is analoog aan het scenario REGISTRATIE. De velden zijn in dit geval ingevuld met reeds aanwezige accountgegevens; de username van een gebruiker kan niet gewijzigd worden. 

\vspace{2 mm}
\hrule
\vspace{4 mm}

\noindent \textbf{Post-conditie}: G krijgt een melding van succesvolle aanpassing. \\



%%%%%% TAALKEUZE

\subsubsection{TAALKEUZE}
\vspace{2 mm}

\textbf{ID}: FR-U006
\vspace{2 mm}

\noindent \textbf{Betrokkenen}: bezoeker (B) of aangemelde gebruiker (G); systeem (S) 
\vspace{2 mm}

\noindent \textbf{Samenvatting}: De interface kan in twee talen, Engels en Nederlands, aangeboden worden. 
\vspace{2 mm}

\noindent \textbf{Pre-conditie}: Indien G, dan wordt de interface aangeboden volgens de taalvoorkeur van de gebruiker, meegegeven bij registratie.
\vspace{2 mm}

\noindent \textbf{Trigger}: G/B geeft aan de taal te willen wijzigen naar Engels of Nederlands (button)

\vspace{4 mm}
\hrule
\vspace{2 mm}
\noindent \textbf{Scenario}:
\noindent S laadt een nieuwe interface waarin de taal aangepast is.
\vspace{2 mm}
\hrule
\vspace{4 mm}

\noindent \textbf{Post-conditie}: G/B ziet een interface in de door hem/haar gekozen taal. \\




%%%%%% 

\subsubsection{BASIS AUTEURSNETWERK}
\vspace{2 mm}

\textbf{ID}: FR-U007
\vspace{2 mm}

\noindent \textbf{Betrokkenen}: aangemelde gebruiker (G); systeem (S) 
\vspace{2 mm}

\noindent \textbf{Samenvatting}: Een gebruiker kan zijn/haar netwerk van co-auteurs raadplegen in een gevisualiseerde graaf. 
\vspace{2 mm}

\noindent \textbf{Pre-conditie}: G is aangemeld en bevindt zich op het dashboard.
\vspace{2 mm}

\noindent \textbf{Trigger}: G geeft aan zijn/haar netwerk te willen weergeven. 

\vspace{4 mm}
\hrule
\vspace{2 mm}
\noindent \textbf{Scenario}:
\begin{enumerate}
\item S genereert een graaf van co-auteurs aan de hand van informatie in de database
\item G kan deze graaf raadplegen en krijgt na het klikken op een node in de graaf basisinformatie over de aangeklikte auteur te zien \textit{[welke info precies wordt weergegeven wordt in volgende iteratie vastgelegd, na overleg aangaande de geplande feature 'userprofile']} 
\end{enumerate}
\vspace{2 mm}
\hrule
\vspace{4 mm}

\noindent \textbf{Post-conditie}: /. \\






\clearpage


%%%%%%%%%%%%%%%%PUBLICATIE%%%%%%%%%%%%%%%%%%%%


\subsection{Functionele Requirements: PUBLICATIE}
\vspace{4 mm}

%%%%%%%%%%%% PUBLICATIE UPLOADEN 
\subsubsection{PUBLICATIE UPLOADEN}
\vspace{2 mm}

\textbf{ID}: FR-P001
\vspace{2 mm}

\noindent \textbf{Betrokkenen}: aangemelde gebruiker (G); systeem (S) 
\vspace{2 mm}

\noindent \textbf{Samenvatting}: G kan een bestand uploaden in PDF- of Bibtex-formaat, waarna S metadata uit dit bestand extraheert; de gevonden metadata kan manueel aangevuld worden door de gebruiker. Het bestand en metadata worden opgeslagen door S. 
\vspace{2 mm}

\noindent \textbf{Pre-conditie}: G bevindt zich op persoonlijk dashboard.  
\vspace{2 mm}

\noindent \textbf{Trigger}: G geeft aan een publicatie te willen uploaden (button)
\vspace{4 mm}



\hrule
\vspace{2 mm}
\noindent \textbf{Scenario}:
\begin{enumerate}
\item G geeft de titel en het type van de publicatie in 
\item S controleert of publicatie met die titel reeds aanwezig is.  $\hookrightarrow$ A
\item G krijgt de mogelijkheid om bestand uit eigen filesystem toe te voegen.
\item S controleert of bestand PDF formaat heeft $\hookrightarrow$ B
\item S vult waar mogelijk volgende metadata aan:
\begin{description}
  \item[titel (V)]  \hfill 
  \item[auteurs (V) ] \hfill 
  \item[indien journal: naam/volume/nummer (V)] \hfill 
  \item[indien proceeding: naam/organisatie (V)] \hfill 
  \item[jaar van publicatie (V) ] \hfill 
  \item[onderzoeksdomein (V)]  \hfill 
  \item[keywords]  \hfill 
  \item[abstract] \hfill 
   \item[aantal citaties] \hfill 
    \item[URL] \hfill 
  \end{description}
\item S controleert of de door scraping gevonden auteurs reeds aanwezig zijn in het systeem. 
\item G kan deze metadata verder aanvullen of wijzigen, en kan oordelen of de in het systeem gevonden auteurs de gewenste auteurs zijn (adhv publicaties van de gevonden auteurs) 
 \item G kan deze informatie doorvoeren (button)  $\hookrightarrow$ C
 \item S slaat publicatie op   
 \item B krijgt melding van succesvol toevoegen publicatie 
\end{enumerate}
\noindent \textbf{Alternatief verloop}:
\begin{description}
\item[ $\hookrightarrow$ A] Een publicatie met dezelfde titel is reeds aanwezig in het systeem. G oordeelt of het over dezelfde publicatie gaar, indien ja, dan wordt deze publicatie aan de bibliotheek van de gebruiker toegevoegd. 
\item[ $\hookrightarrow$ B] Het bestand heeft niet het juiste formaat. G krijgt een melding. $\hookrightarrow$ 1.
\item[ $\hookrightarrow$ C] De ingevoerde metadata voldoet niet aan de validatie-vereisten. G krijgt een melding en kan aanpassingen doorvoeren. $\hookrightarrow$ 5.
\end{description}

\vspace{2 mm}
\hrule
\vspace{4 mm}


\noindent \textbf{Post-conditie}: Een nieuwe publicatie met bijhorende metadata is opgeslagen in  de bibliotheek van de gebruiker. \\



%%%%%%%%%%%% OPZOEKEN EN BEWAREN VAN PUBLICATIE
\subsubsection{OPZOEKEN EN BEWAREN VAN PUBLICATIE}
\vspace{2 mm}

\textbf{ID}: FR-P002
\vspace{2 mm}

\noindent \textbf{Betrokkenen}: aangemelde gebruiker (G); systeem (S) 
\vspace{2 mm}

\noindent \textbf{Samenvatting}: G kan zoeken naar een publicatie op twee verschillende manieren: door het invoeren van een generieke zoekterm (=basic search), waarna resultaten verzameld worden zowel in het Skribl systeem als via google scholar; of via een uitgebreide zoekopdracht, waarna publicaties enkel gezocht worden binnen het Skribl systeem. G kan de gevonden publicaties toevoegen aan zijn/haar gekozen bibliotheek. 
\vspace{2 mm}

\noindent \textbf{Pre-conditie}: G bevindt zich in bibliotheek. 
\vspace{2 mm}

\noindent \textbf{Trigger}: G geeft aan publicaties te willen zoeken (form en button)
\vspace{4 mm}

\hrule
\vspace{2 mm}
\noindent \textbf{Scenario}:
\begin{enumerate}
\item G vult een generieke zoekterm in, of geeft aan een uitgebreide zoekopdracht te willen uitvoeren met als mogelijke parameters titel, jaar, keywords en auteurs.
\item G klikt aan de zoekopdracht te willen uitvoeren (button)
\item S doorzoekt eigen systeem van openbare publicaties en verzamelt relevante publicaties
\item indien basis zoekopdracht, dan contacteert S Google Scholar en verzamelt relevante publicaties
\item G ziet een lijst van de gevonden publicaties en kan de metadata (inclusief abstract, indien aanwezig) inspecteren
\item G kan de PDF openen of de URL naar de gevonden PDF openen 
\item G kan ervoor kiezen de publicatie (PDF/URL + metadata) in zijn bibliotheek of portfolio op te slaan
\end{enumerate}

\vspace{2 mm}
\hrule
\vspace{4 mm}

\noindent \textbf{Post-conditie}: Een nieuw bestand (PDF of URL) met bijhorende metadata is opgeslagen in het systeem / de bibliotheek van de gebruiker. \\


\subsubsection{BIBLIOTHEEK}
\vspace{2 mm}

\textbf{ID}: FR-P003
\vspace{2 mm}

\noindent \textbf{Betrokkenen}: aangemelde gebruiker (G); systeem (S) 
\vspace{2 mm}

\noindent \textbf{Samenvatting}: G kan publicaties toevoegen en beheren in 'bibliotheken'. De standaard bibliotheken zijn: alle bewaarde publicaties, een portfolio (bestaande uit eigen publicaties), en favorieten (bestaande uit bewaarde publicaties van derden). G kan hier bliotheken aan toevoegen. 
\vspace{2 mm}

\noindent \textbf{Pre-conditie}: G bevindt zich in bibliotheek. 
\vspace{2 mm}

\noindent \textbf{Trigger}: / 
\vspace{4 mm}

\hrule
\vspace{2 mm}
\noindent \textbf{Scenario}:
\begin{enumerate}
\item portfolio: G kan een lijst van eigen publicaties beheren 
\item favorieten: G kan een lijst van andere publicaties beheren 
\item alle bewaarde publicaties: G kan een lijst van al zijn bewaarde publicaties (portfolio + favorieten) beheren 
\item G kan een nieuwe bibliotheek aanmaken
\item G kan in elk van deze bibliotheken een publicatie toevoegen, via een zoekopdracht of vanuit het eigen systeem 
\item G kan in elk van deze bibliotheken een publicatie verwijderen 
\item G kan aangeven dat een publicatie privaat is 
\item G krijgt na selectie van een publicatie de metadata van deze publicatie te zien
\item G kan de PDF openen of kan de link naar de PDF volgen 
\end{enumerate}

\vspace{2 mm}
\hrule
\vspace{4 mm}


\noindent \textbf{Post-conditie}: / \\


\subsubsection{PDF-VIEWER}
\vspace{2 mm}

\textbf{ID}: FR-P004
\vspace{2 mm}

\noindent \textbf{Betrokkenen}: aangemelde gebruiker (G); systeem (S) 
\vspace{2 mm}

\noindent \textbf{Samenvatting}: G kan publicaties bekijken met behulp van een ingebouwde PDF viewer. \vspace{2 mm}

\noindent \textbf{Pre-conditie}: G is aangemeld 
\vspace{2 mm}

\noindent \textbf{Trigger}: G klikt op een publicatie die hij/zij wil bekijken 
\vspace{4 mm}

\hrule
\vspace{2 mm}
\noindent \textbf{Scenario}:
\begin{enumerate}
\item  G kan de publicatie bekijken in een viewer
\item G kan inzoomen, uitzoomen en door het document bladeren 
\end{enumerate}
\vspace{2 mm}
\hrule
\vspace{4 mm}


\noindent \textbf{Post-conditie}: / \\




\clearpage


%%%%%%%%%%%%%%%%DATA -MINING%%%%%%%%%%%%%%%%%%%%


\subsection{Functionele Requirements: DATA-MINING}
\vspace{4 mm}


\subsubsection{GOOGLE SCHOLAR SCRAPING}
\vspace{2 mm}

\textbf{ID}: FR-D001
\vspace{2 mm}

\noindent \textbf{Betrokkenen}: systeem (S), Google Scholar (GS)
\vspace{2 mm}

\noindent \textbf{Samenvatting}: S kan gegevens en metadata verzamelen via http requests naar Google Scholar  
\vspace{2 mm}

\noindent \textbf{Pre-conditie}: /
\vspace{2 mm}

\noindent \textbf{Trigger}: Gebruiker voegt nieuwe paper toe en wil metadata opvragen, gebruiker voert zoekopdracht uit, etc.
\vspace{4 mm}

\hrule
\vspace{2 mm}
\noindent \textbf{Scenario}:
\begin{enumerate}
\item S kan volgende, mogelijk onvolledige, metadata verzamelen via een one-form GS zoekopdracht 
\begin{description}
  \item[titel]  \hfill
  \item[auteurs] \hfill 
   \item[abstract] \hfill
   \item[titel van journal of proceedings ] \hfill 
    \item[uitgever] \hfill 
   \item[jaar van publicatie] \hfill 
    \item[aantal citaties] \hfill
   \item[URL] \hfill 
  \end{description}
\end{enumerate}

\vspace{2 mm}
\hrule
\vspace{4 mm}


\noindent \textbf{Post-conditie}: / \\


\clearpage





