
\subsubsection{algemene risico's}

\subsubsection*{Een te grote werklast}
\begin{itemize}
\item waarschijnlijkheid: groot
\item melding: Groepsleden geven aan wanneer de werklast voor hen te groot dreigt te worden.
\item oplossing: Aan het begin van iedere sprint worden de opdrachten geprioriteerd. Indien de werkdruk te hoog zou worden kunnen minder belangrijke opdrachten weggelaten of opgeschoven worden. De laatste sprint laat ruimte voor het uitvoeren van eventueel onafgewerkte features. Tijdens blok- en examenperiodes en tijdens vakanties wordt er niet verwacht van de groepsleden aan het project te werken.
\end{itemize}

\subsubsection*{Onvoldoende ervaring of kennis}
\begin{itemize}
\item waarschijnlijkheid: groot
\item melding: Groepsleden geven duidelijk aan wanneer zij het gevoel hebben dat het hen aan de nodige kennis of ervaring ontbreekt. De Project Manager maakt ruimte in de planning voor het wegwerken van de achterstand (indien mogelijk) en geeft dit ook expliciet als taak op.
\item oplossing: Het herhalen van cursussen en/of het volgen van online tutorials.
\end{itemize}

\subsubsection*{Afwezigheid of ziekte}
\begin{itemize}
\item waarschijnlijkheid: groot
\item melding: Groepsleden laten zo vroeg mogelijk weten wanneer en hoelang zij afwezig zullen zijn.
\item oplossing: Voor iedere rol is er een persoon aangeduid als back-up. Deze volgt de activiteiten van de hoofdverantwoordelijke op en kan deze rol overnemen indien nodig.
\end{itemize}

\subsubsection*{Eindresulaat is niet gebruiksvriendelijk}
\begin{itemize}
\item waarschijnlijkheid: gemiddeld
\item melding: De beta versies zullen worden voorgelegd aan enkele externe testers, zoals bereidwillige assistenten, maar ook personen die niet geaffilieerd zijn aan de wetenschappelijk wereld.
\item oplossing: De feedback van deze testers wordt gebruikt om aanpassingen aan het product door te voeren
\end{itemize}

\subsubsection{projectgebonden risico's}

\subsubsection*{vorm (meta)data niet voldoende consistent}
\begin{itemize}
\item waarschijnlijkheid: groot
\item beschrijving: de door gebruikers ingevoerde data is niet van een consistente vorm, e.g. de ene gebruiker geeft als universiteit 'VUB', de andere 'Vrije Universiteit Brussel'; de ene geeft auteurs met initialen, de andere met voornamen voluit etc. Dit bemoeilijkt het vinden van relaties om het netwerk van een gebruiker op te bouwen.
\item oplossing: Waar mogelijk het verwachte formaat specificeren of de keuzemogelijkheden beperken; eventueel reeds in het systeem aanwezige data als optie voorstellen.
\end{itemize}

