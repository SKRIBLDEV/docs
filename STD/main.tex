\documentclass{article}
\usepackage[utf8]{inputenc}
\usepackage{natbib}
\usepackage{graphicx}
\usepackage{lastpage}
\usepackage{xcolor}
\usepackage{lipsum}
\usepackage[T1]{fontenc}
\usepackage{fancyhdr}
\usepackage{geometry}
\usepackage{url}
\usepackage[dutch]{babel}
\usepackage{footnote}



\bibliographystyle{abbrv} %abbrvnat geeft problemen

\title{Software Test Plan}
\author{} %leave empty
\date{20 april 2015} %ok, manuele datum

\addtolength{\footskip}{1.3cm} % make more space for the footer
\pagestyle{fancyplain} % use fancy for all pages except chapter start
\lhead{}
\cfoot{\includegraphics[height=1.3cm]{Small_Logo.png}} % right logo
\rfoot{\thepage}
\renewcommand{\headrulewidth}{0.3pt} % remove rule below header
\renewcommand{\footrulewidth}{0.3pt} % remove rule below header

\begin{document}

\makeatletter
\begin{titlepage}

\newcommand{\HRule}{\rule{\linewidth}{0.7mm}} % Defines a new command for the horizontal lines, change thickness here


\vspace*{1.2mm}

\center 
\includegraphics[scale=0.6]{Logo.png}\\[1cm] 
%---------------------------------------------------------------------------------------
%	HEADINGS SECTION
%----------------------------------------------------------------------------------------

\textsc{\LARGE Vrije Universiteit Brussel}\\[0.3cm] % Name of your university/college
\textsc{\large WE-DINF-6537}\\
\textsc{\large Project Software Engineering}\\
\textsc{\large Academiejaar 2014-2015}\\[0.3cm] 
%\textsc{\large Software Engineering}\\[0.7cm] % Major heading such as course name

%----------------------------------------------------------------------------------------
%	TITLE SECTION
%----------------------------------------------------------------------------------------

\HRule \\[0.4cm]
{ \huge \bfseries \@title \\[0.5cm] }
\HRule \\[0.5cm]
 
%----------------------------------------------------------------------------------------
%	AUTHOR SECTION
%----------------------------------------------------------------------------------------

\Large
% laat voorlopig nog even de namen/mailadressen op deze pagina staan
% eerst moeten we zien of er een beter alternatief is om de lege plek dan op te vullen
% anders laten we ze gewoon staan...

%volgens mij ziet het er zo heel goed uit, maar als ze echt wegmoeten mss vub logo groter maken? 
% => OK, ik vind het ook  beter zo, we laten ze staan!
Douglas Horemans \textit{<dhoreman@vub.ac.be>}\\
Hannah Pinson \textit{<hpinson@vub.ac.be>}\\
Ivo Vervlimmeren \textit{<ivervlim@vub.ac.be>}\\
Noah Van Es \textit{<noahves@vub.ac.be>}\\
Pieter Steyaert \textit{<psteyaer@vub.ac.be>}\\

\vspace{0.6cm}

\includegraphics[scale=0.4]{VUB_schild.pdf}\\[0.5cm]

{\large 19 november 2014}
\vfill % Fill the rest of the page with whitespace

\end{titlepage}

\newpage
\section*{Versiegeschiedenis}
\addcontentsline{toc}{section}{Versiegeschiedenis}

\begin{center}
\begin{tabular}[t]{| c | c | c | c |}
    \hline
    \textbf{Versie} & \textbf{Datum} & \textbf{Auteurs\cite{note:author}} & \textbf{Beschrijving} \\
    \hline
    
    1.0     &  05/11/2014   &   \begin{tabular}{c} 
                                    Hannah Pinson \\
                                    Ivo Vervlimmeren \\
                                    Noah Van Es \\ 
                                    Pieter Steyaert \\
                                \end{tabular} & Aanmaak eerste versie \\
    \hline
    1.1     &  18/11/2014   &   \begin{tabular}{c} 
                                    Hannah Pinson \\
                                     \end{tabular} & \begin{tabular}{c}  
                                     			Gekregen feed-back  gedeeltelijk doorgevoerd\\
                                     			planning eerste sprint aangevuld \\
							 \end{tabular}  \\
    \hline

\end{tabular}
\end{center}
\newpage

\tableofcontents
\newpage


\section{Introductie}
SKRIBL is een sociaal netwerk voor proffen en PHD studenten.
\\
Dit document is een testplan voor de SKRIBL web-application.  
Het testplan beschrijft hoe de code getest moet worden opdat alle functionaliteiten van Skribl perfect zouden werken.
\\
Vermits het SKRIBL-team agile werkt, zal dit testplan nog sterk veranderen doorheen de iteraties. Elke test heeft een testversie omdat de tests met de iteraties mee moeten evolueren. Agile werken houdt ook in dat de vorige geslaagde tests opnieuw getest  worden en indien nodig aangepast worden.

\section{Referenties}
\begingroup
\renewcommand{\section}[2]{}% verwijdert titel sectie referenties
\bibliography{referenties}
\endgroup
%kan je voor onderliggende items \cite{naam1:naam2} gebruiken? alles wordt dan automatisch opgelijst door bovenliggend gedeelte.
\begin{itemize}
  \item SDD versie 1.0
  \item Internal manual \url{http://wilma.vub.ac.be/~se4_1415/internalManuals/test-framework.pdf}
  \item Jasmine \url{http://jasmine.github.io/2.0/introduction.html}
  \item Postman \url{https://www.getpostman.com/}
  \item Selenium \url{http://www.seleniumhq.org/}
  \item IEEE 829
\end{itemize}


\section{Definities}
\begin{description}
\item[HTML]Hyper Text Markup Language. Is een taal dat de inhoud van een webpagina beschrijft.
\item[DOM]Document Object Model. Is een soort boom dat de content van een webpagine bevat.
\item[GUI]Graphical User Interface.
\item[API]Application Programming Interface.
\item[MVC]Model View Controller. Dit is een software architectuur patroon dat de implementatie opsplitst in drie delen: Model, View en Controller. De controllers worden gebruikt om het model aan te passen, die op hun beurt de view aanpassen. 
\end{description}

\section{Domein}
In het SDD, sectie 3.2, wordt de architectuur van de SKRIBL web-application beschreven. De sectie beschrijft de drie tiers die de webapplicatie gebruikt, namelijk de databasetier, servertier en clienttier. Voor deze drie tiers moeten tests worden ge\"implementeerd, en deze worden in de sectie "Test items" beschreven.

\section{Test items}
Alle ge\"implementeerde items \& features moeten getest worden. De tests kunnen in verschillende groepen onderverdeeld worden. Deze groepen worden verderop beschreven.

\subsection{Databasetier tests}
De database houdt alle gegevens bij die te maken hebben met users, publicaties,... In deze tier moet het toevoegen, opvragen en verwijderen van die gegevens getest worden. Het is ook belangrijk dat na het manipuleren van de database deze niet corrupt wordt.
\\
\\
Alle tests bevinden zich in : $\backslash$server$\backslash$server$\backslash$modules$\backslash$database$\backslash$jasmine-tests$\backslash$...
\subsubsection{Data-toevoeging tests}
Tests die controleren of data op de juiste manier wordt toegevoegd en opgeslagen in de database.
\begin{itemize}
\item test adding user
\item test adding journal 
\item test adding proceeding
\item test adding library to user
\item test adding publication to library
\item test updating publication
\end{itemize}
File : databaseSetupSpecs.js
\subsubsection{Data-opvraging tests}
Tests die controleren of data op de juiste manier wordt opgehaald uit de database.
\begin{itemize}
\item test opvragen gebruiker
\item test opvragen pub data
\item test opvragen uploader pub
\item test opvragen bibliotheken van gebruiker
\item test opvragen pubs uit bibliotheek
\item test simpel zoeken naar pub
\item test geavanceerd zoeken naar pub
\item test opvragen pubs van auteur
\item test zoeken auteur met naam ...
\end{itemize}
File : databaseRequestsSpecs.js
\subsubsection{Data-verwijdering tests}
Tests die controleren of data op de juiste manier verwijderd wordt. Het verwijderen van bijvoorbeeld een user moet meerdere elementen in de database verwijderen, ook dit wordt gecontroleerd. 
\begin{itemize}
\item test verwijder publicatie uit bibliotheek
\item test verwijder bibliotheek van gebruiker
\item test verwijder journal
\item test verwijder proceeding
\item test verwijder user
\end{itemize}
File : databaseDeletionSpecs.js
\subsubsection{Data-persistentie tests}
Tests die nagaan of data wel daadwerkelijk bewaard blijft en de database niet corrupt is.
\begin{itemize}
\item test user data
\item test author data
\item test library data
\item test journal data
\item test proceeding data
\item test affiliation data
\item test researchdomains
\item test keywords
\end{itemize}
File : databasePersistenceSpecs.js
\subsection{Servertier tests}
De servertier is een communicatielaag tussen de database en de user. Voor deze tier moeten alle HTTP requests getest worden. Ook dient het scrapen en de regex expressies die toegepast worden op de userinput getest te worden. 
\\\\
Alle HTTP tests bevinden zich in : $\backslash$server$\backslash$server$\backslash$routeTests$\backslash$...\\
Scraping en userinput tests bevinden zich in : $\backslash$server$\backslash$server$\backslash$modules$\backslash$tests$\backslash$...
\subsubsection{Authors tests, HTTP}
De tests omtrent auteurs dienen enerzijds om te zoeken naar auteurs op naam en voornaam, anderzijds om ook extra informatie van een bepaalde auteur op te laden. Zo wordt bijvoorbeeld de visualisatie van de co-auteurrelaties in een graph hier ook getest\\\\
File : authors.json
\subsubsection{Library tests, HTTP}
Hierin vindt men testscenario's terug om bibliotheken te beheren. Dit omvat zowel het aanmaken, laden en verwijderen van de bibliotheken als het toevoegen (en verwijderen) van publicaties in deze bibliotheken.\\\\
File : libraries.json
\subsubsection{Publications tests, HTTP}
Publicaties vormen een cruciaal onderdeel van de webapplicatie en dus ook van de server API. Eerst en vooral wordt hier het uploaden, aanpassen, downloaden en verwijderen van een publicatie (en haar metadata) nagekeken. Verder worden ook meer geadvanceerde functionaliteiten zoals het extraheren van metadata en het zoeken naar publicaties (zowel intern als extern) hier getest.\\\\
File : publications.json
\subsubsection{Users test, HTTP}
De belangrijkste tests omtrent gebruikers omvatten het inloggen en registreren.
Daarnaast wordt ook het laden en verwijderen van een gebruiker getest.\\\\
File : users.json
\subsubsection{Scarping tests}
Tests dat nagaan of scraping op de juiste manier werkt.\\\\
File : scrapingSpec.js
\subsubsection{Userinput tests}
Tests dat nagaan of regex expressies die op userinput worden toegepast juist werken. \\\\
File: userSpec.js + validationSpec.js
\subsection{Clienttier}
De clienttier zorgt voor een user interface en de communicatie met de server. Deze tier wordt getest aan de hand van de GUI.\\
Om tijdsverlies te voorkomen wordt de GUI zoveel mogelijk automatisch getest. Sommige delen zijn niet automatisch testbaar en dienen manueel getest te worden. Natuurlijk moet de tester tijdens het automatisch testen nog steeds bekijken of alles op de juiste manier verloopt. Informatie over het automatisch testen van de GUI is beschikbaar in sectie "Strategie".\\
\begin{description}
\item[OPGEPAST] automatische GUI testing betekent NIET dat de tester niet aanwezig moet zijn om na te kijken dat alles op een juiste manier verloopt. Automatisch GUI testing is niet even precies als unit testing.\\
Bijvoorbeeld: een GUI test kan slagen maar visueel problemen veroorzaken die de tester wel ziet en de automatische GUI tester niet.
\end{description}

\begin{description}
\item[A] Automatische GUI testing
\item[M] Manuele GUI testing
\end{description}
\subsubsection{User tests}
\begin{itemize}
  \item Login test A
  \item Logout test A
  \item Register test A
  \item Delete user test A
\end{itemize}
\subsubsection{Library managment tests}
\begin{itemize}
  \item Publicatie toevoegen via search test A
  \item Libraries management test A
  \item Downloaden van publicaties test M
  \item Informatie over publicaties aanpassen test M
  \item Informatie over publicaties opvragen test A
\end{itemize}
\subsubsection{Upload test}
\begin{itemize}
  \item Publicaties toevoegen(via upload) test A
\end{itemize}
\subsubsection{Search Tests}
\begin{itemize}
\item Search test A
\item Advanced search test A
\end{itemize}
\subsubsection{Network}
\begin{itemize}
\item Show network A
\item Load other user network M
\end{itemize}
\subsubsection{Preview}
\begin{itemize}
\item Show preview A
\item Control preview M
\end{itemize}
\subsubsection{Recommender}
\begin{itemize}
\item Recommender systeem M
\end{itemize}
\section{Software risico gevallen}
De client die de SKRIBL web-application gebruikt komt alleen maar in aanraking met de clienttier. Opdat deze tier zonder probleem zou werken, moeten de servertier en database tier ook werken. Dit impliceert dat een fout in de server of database tier heel de web-application kan verzwakken. Natuurlijk zijn problemen in de web-application voorzien en kan SKRIBL met de meeste problemen leven, als bijvoorbeeld een http request faalt zal heel de application niet crashen. Maar om toch zo weinig mogelijk problemen te hebben is het testen  van ELKE tier zeer belangrijk. Het grootste risico van onze applicatie is dus het feit dat de onderliggende tiers goed genoeg moeten werken zodat de client de application op een zo goed mogelijke manier kan gebruiken.
%
\section{Features}
De features die getest worden zijn:
\begin{itemize}
  \item Login 
  \item Logout 
  \item Register
  \item Delete user
  \item Publicaties toevoegen(via upload)
  \item Publicatie opzoeken(zowel internal als external) om publicatie toe te voegen
  \item Libraries managen
  \item Downloaden van publicaties
  \item Informatie over publicaties aanpassen
  \item Informatie over publicaties opvragen
  \item Auteur Network
  \item Recommender systeem
  \item Pdf Viewer
\end{itemize}
Sommige features die getest worden houden andere features in (die ook getest moeten worden) en worden hier onder beschreven.\\
\begin{description}
\item[Publicaties toevoegen(via upload)] = publicatie zoeken + scraping + informatie over publicaties aanpassen + publicatie toevoegen aan library
\item[Libraries managen] = user libraries ophalen + publicaties van een library ophalen + publicatie toevoegen aan library + publicatie uit library verwijderen + library aanmaken + library verwijderen.
\end{description}
%
\section{Strategie}
Om de code te testen worden drie frameworks gebruikt: 
\begin{description}
  \item[Jasmine 2.0.3] Jasmine is een javascript test framework met heel veel kracht.
  \item[Postman] Een chrome extension dat HTTP requests kan versturen. 
  \item[Selenium] Een geautomatiseerde GUI tester.
\end{description}
 %
\subsection{Jasmine 2.0.3}
Om code te testen in Jasmine moet een specification file worden aangemaakt. Deze specification file beschrijft de tests die dienen uitgevoerd te worden. 
\newline
Om tests te beschrijven worden matchers gebruikt. Matchers kunnen elementen vergelijken, controleren of het element wel gedefinieerd is en nog veel anderen! Het is ook mogelijk om zelf matchers te schrijven.
Voorbeelden van matchers :
\begin{itemize}
  \item expect().toBe();
  \item expect().toEqual();
  \item expect().toMatch();
  \item ...
\end{itemize}
%
\subsubsection{Tests uitvoeren}
Omdat Jasmine een standalone software is, is het uitvoeren van tests heel gemakkelijk. Wanneer een specification bestand geschreven is moet het pad naar dit bestand alleen maar toegevoegd worden in een html bestand. Wanneer dit gedaan is dient het html bestand alleen maar geopend te worden in een webbrowser. In de webbrowser geeft Jasmine de geslaagde en gefaalde tests weer.
%
\subsubsection{Extra}
Een manual over Jasmine werd opgesteld door de test manager voor het SKRIBL-team. Deze staat in de referenties en legt het gebruik van Jasmine in detail uit.
%
\subsection{Postman}
Postman laat toe om op een heel simpele manier HTTP requests samen te stellen en vervolgens op te sturen aan de hand van een GUI. Het is daarna het werk van de tester om te kijken of de teruggegeven status de verwachte is of niet. In principe zijn alle HTTP responses tussen 200 en 299 juist.
%
\subsubsection{Tests uitvoeren}
Postman laat toe om de HTTP requests te schrijven en deze te bewaren. Het is hierna simpel om de HTTP requests (tests) later in te laden in postman om alle requests opnieuw te testen.
%
\subsection{Selenium}
Selenium laat toe om GUI testing te reproduceren. Selenium kan op knoppen klikken, veldjes invullen enz. Ook laat het toe om delen van de GUI te vergelijken met bepaalde waarden. 
%
\subsubsection{Tests uitvoeren}
Het is mogelijk om tests te maken en deze een per een uit te voeren. Ook kunnen de tests opgeslagen worden en later weer ge\"importeerd worden en opnieuw uitgevoerd worden. 
%
\section{Omgeving}
Jasmine heeft geen bepaalde omgeving nodig om tests te kunnen uitvoeren . Het enige wat gedaan moet worden is de library uitbreiden met de gewenste source en specification bestanden.
\\
Daarentegen heeft Postman Google Chrome nodig, en Selenium Mozilla Firefox, vermits het allebei extensions zijn.
%
\section{Verantwoordelijkheden}
In het SKRIBL-team worden meerdere programmeertalen en libraries gebruikt afhankelijk van de tier. Elk lid van het SKRIBL team werkt aan een bepaalde tier en kent de libraries en programmeertalen die gebruikt worden voor die tier. Ook kent elk lid natuurlijk de code van zijn tier maar niet alle code van elke tier. Om al deze redenen schrijft elke team lid zijn eigen tests.\\
Wanneer de code van een lid aan al zijn tests voldoet en alle gevraagde functionaliteiten zijn geprogrammeerd (en getest a.d.h.v zelf geschreven test) controleert de test manager alle tests. De test manager kan dan eventueel vragen om tests toe te voegen indien hij vindt dat er voor sommige functionaliteiten meer nodig zijn, of dat er dieper in de code moet getest worden, enz.
%
\section{Goedkeuring}
Wanneer de Test Manager alle tests heeft goedgekeurd zoals beschreven in sectie "Verantwoordelijkheden", moet de Quality Assurance Manager de code bekijken en deze wel of niet valideren. De eisen van de QA worden in volgende sectie besproken. Indien de Quality Assurance Manager de code niet valideert moet de programmeur zijn code aanpassen tot de code aan de eisen van de Test Manager en Quality Assurance Manager voldoet. 
\\
Wanneer code wordt gevalideerd kan deze gepushed worden naar de master branch van de SKRIBL repository.
%
\section{Instructies van de QA}
De QA is verantwoordelijk voor het bewaken van de kwaliteit van afgeleverde code (documenten daarentegen worden nagekeken door de document master). Dit houdt in dat na het schrijven en testen van de code een extra validatiestap zal plaatsvinden waarbij er niet wordt gekeken naar de functionaliteit zelf (dit gebeurt namelijk bij het testen), maar wel naar de leesbaarheid, performantie en onderhoudbaarheid van de code zoals gedefinieerd door enkele standaard definities. 
\\\\
Om dit proces te sturen werden enkele richtlijnen opgesteld. Deze omvatten onder andere het hanteren van een zekere stijl en het vermijden van minder idiomatische kenmerken van JavaScript. Daarnaast worden enkele metrieken bijgehouden zoals:\\
\begin{itemize}
\item LOC per klasse/functie/module
\item nuttige hoeveelheid commentaren (hiervoor wordt tevens gebruik gemaakt van de JSDoc-standaard, cf. SPMP)
\item aantal geneste if/for/…-lussen
\item ...
\end{itemize}
Hiervoor wordt intern gebruik gemaakt van CodeClimate, een tool die in staat is om JavaScript code te analyseren en vervolgens enkele vaak voorkomende pitfalls (zoals duplicatie) er uit te halen. Hoewel deze score niet representatief is voor de uiteindelijke codekwaliteit, is het een handige indicator om in een snel tempo de meeste prominente stijlfouten er uit te halen. Elk groepslid wordt dan ook verondersteld om zijn/haar rapport van CodeClimate te analyseren en vervolgens bij te werken waar nodig.
\\\\
Ten slotte zal de QA aan de hand van een steekproef bepaalde delen van de code zelf bekijken en de nodige feedback waar nodig. Opnieuw wordt hierbij vooral gekeken naar de codestijl en wordt bij het tegengekomen van performantieproblemen de desbetreffende code onderworpen aan profiling (hetzij door de QA, hetzij door developer zelf). De bedoeling is dat de persoon in kwestie uit deze richtlijnen en commentaren dan de nodige principes kan halen om zijn/haar hele codebase bij te werken. Van zodra dit is gebeurd, kan de QA indien nodig dan nog overgaan tot een finale analyse van alle op te leveren code, waarbij opnieuw een gelijkaardige werkwijze en criteria worden gehanteerd.
%
\clearpage
%
\end{document}
