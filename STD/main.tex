\documentclass{article}
\usepackage[utf8]{inputenc}
\usepackage{natbib}
\usepackage{graphicx}
\usepackage{lastpage}
\usepackage{xcolor}
\usepackage{lipsum}
\usepackage[T1]{fontenc}
\usepackage{fancyhdr}
\usepackage{geometry}
\usepackage{url}
\usepackage[dutch]{babel}



\bibliographystyle{abbrv} %abbrvnat geeft problemen

\title{Software Test Plan}
\author{} %leave empty
\date{19 november 2014} %ok, manuele datum

\addtolength{\footskip}{1.3cm} % make more space for the footer
\pagestyle{fancyplain} % use fancy for all pages except chapter start
\lhead{}
\cfoot{\includegraphics[height=1.3cm]{Small_Logo.png}} % right logo
\rfoot{\thepage}
\renewcommand{\headrulewidth}{0.3pt} % remove rule below header
\renewcommand{\footrulewidth}{0.3pt} % remove rule below header

\begin{document}

\makeatletter
\begin{titlepage}

\newcommand{\HRule}{\rule{\linewidth}{0.7mm}} % Defines a new command for the horizontal lines, change thickness here


\vspace*{1.2mm}

\center 
\includegraphics[scale=0.6]{Logo.png}\\[1cm] 
%---------------------------------------------------------------------------------------
%	HEADINGS SECTION
%----------------------------------------------------------------------------------------

\textsc{\LARGE Vrije Universiteit Brussel}\\[0.3cm] % Name of your university/college
\textsc{\large WE-DINF-6537}\\
\textsc{\large Project Software Engineering}\\
\textsc{\large Academiejaar 2014-2015}\\[0.3cm] 
%\textsc{\large Software Engineering}\\[0.7cm] % Major heading such as course name

%----------------------------------------------------------------------------------------
%	TITLE SECTION
%----------------------------------------------------------------------------------------

\HRule \\[0.4cm]
{ \huge \bfseries \@title \\[0.5cm] }
\HRule \\[0.5cm]
 
%----------------------------------------------------------------------------------------
%	AUTHOR SECTION
%----------------------------------------------------------------------------------------

\Large
% laat voorlopig nog even de namen/mailadressen op deze pagina staan
% eerst moeten we zien of er een beter alternatief is om de lege plek dan op te vullen
% anders laten we ze gewoon staan...

%volgens mij ziet het er zo heel goed uit, maar als ze echt wegmoeten mss vub logo groter maken? 
% => OK, ik vind het ook  beter zo, we laten ze staan!
Douglas Horemans \textit{<dhoreman@vub.ac.be>}\\
Hannah Pinson \textit{<hpinson@vub.ac.be>}\\
Ivo Vervlimmeren \textit{<ivervlim@vub.ac.be>}\\
Noah Van Es \textit{<noahves@vub.ac.be>}\\
Pieter Steyaert \textit{<psteyaer@vub.ac.be>}\\

\vspace{0.6cm}

\includegraphics[scale=0.4]{VUB_schild.pdf}\\[0.5cm]

{\large 19 november 2014}
\vfill % Fill the rest of the page with whitespace

\end{titlepage}

\newpage
\section*{Versiegeschiedenis}
\addcontentsline{toc}{section}{Versiegeschiedenis}

\begin{center}
\begin{tabular}[t]{| c | c | c | c |}
    \hline
    \textbf{Versie} & \textbf{Datum} & \textbf{Auteurs\cite{note:author}} & \textbf{Beschrijving} \\
    \hline
    
    1.0     &  05/11/2014   &   \begin{tabular}{c} 
                                    Hannah Pinson \\
                                    Ivo Vervlimmeren \\
                                    Noah Van Es \\ 
                                    Pieter Steyaert \\
                                \end{tabular} & Aanmaak eerste versie \\
    \hline
    1.1     &  18/11/2014   &   \begin{tabular}{c} 
                                    Hannah Pinson \\
                                     \end{tabular} & \begin{tabular}{c}  
                                     			Gekregen feed-back  gedeeltelijk doorgevoerd\\
                                     			planning eerste sprint aangevuld \\
							 \end{tabular}  \\
    \hline

\end{tabular}
\end{center}
\newpage

\tableofcontents
\newpage


\section{Introductie}
SKRIBL is een sociaal netwerk voor proffen en PHD studenten.
\\
Dit document is een test plan voor de SKRIBL web-application.  
Het test plan beschrijft hoe de code \& features  worden getest.
\\
Vermits het SKRIBL-team agile werkt, gaat dit testplan nog sterk veranderen doorheen de iteraties. Elke test heeft een testversie omdat de tests met de iteraties mee moeten evolueren. Agile werken houdt ook in dat de vorige geslaagde test, opnieuw getest  worden en indien nodig aangepast worden.

\section{Referenties}
\begingroup
\renewcommand{\section}[2]{}% verwijdert titel sectie referenties
\bibliography{referenties}
\endgroup
%kan je voor onderliggende items \cite{naam1:naam2} gebruiken? alles wordt dan automatisch opgelijst door bovenliggend gedeelte.
\begin{itemize}
  \item SDD versie 1.0
  \item Internal manual \url{http://wilma.vub.ac.be/~se4_1415/internalManuals/test-framework.pdf}
  \item Jasmine \url{http://jasmine.github.io/2.0/introduction.html}
  \item IEEE 829
\end{itemize}


\section{Definities}
\begin{description}
\item[HTML]Hyper Text Markup Language. Is een taal dat de inhoud van een webpagina beschrijft.
\item[DOM]Document Object Model. Is een soort boom dat de content van een webpagine bevat.
\item[GUI]Graphical User Interface.
\item[API]Application Programming Interface.
\item[MVC]Model view controller. Is de link tussen de front-end van een web app en de server(in een Model view).
\end{description}

\section{Domein}
In het SDD, sectie 3.2, wordt de architectuur van de SKRIBL web-application beschreven. De sectie beschrijft de drie tiers dat de web-application gebruikt, namelijk de databasetier, servertier en clienttier. Deze drie tiers moeten tests implementeren, die in de sectie "Test items" worden beschreven.

\section{Test items}

Alle ge\"implementeerde items \& features moeten getest worden. De tests kunnen in verschillende groepen tests verdeeld worden. Deze groepen worden verder uitgelegd.

\begin{table}[h]
\begin{tabular}{l | c }
 Item&Test versie\\
 \hline
 Databasetier tests&1.0\\
 Servertier tests&1.0\\
 Clienttier tests test&1.0\\
\end{tabular}
\end{table}

\subsection{Databasetier tests}
De database houdt alle gegevens bij die te maken hebben met users, publicaties,... In deze tier moet het toevoegen, uithalen en verwijderen van die gegevens getest worden. Het is ook belangrijk dat na het manipuleren van de database deze niet corrupt wordt.

\subsubsection{Data-persistentie tests}
Tests dat nagaan of data wel op de juiste manier wordt toegevoegd in de database, en daadwerkelijk bewaard blijven.
\\
\\
%File : 
\subsubsection{Data-opvraging tests}
Tests die controleren of de gevraagde data teruggegeven wordt aan de server tier wanneer deze data opvraagt.
\\
\\
%File :
\subsubsection{Data-verwijdering tests}
Controleren of data op de juiste manier verwijderd wordt. Het verwijderen van bijvoorbeeld een user, moet meerdere elementen in de database verwijderen, dit word gecontroleerd.
\\
\\
%File : 
\subsection{Servertier tests}
De servertier is een soort communicatie laag tussen de database en de user. De server moet HTTP requests behandelen en voor de validatie hiervan zorgen.

\subsubsection{HTTP requests tests}
Probeert juiste en foute HTTP requests te sturen naar de server en bekijkt het antwoord van de server.
\\
\\
%File : 
\subsubsection{Validatie tests}
Deze tests proberen door middel van HTTP requests na te gaan of het valideren van data juist gebeurt.
\\
\\
%File :
\subsection{Clienttier}
De clienttier zorgt voor een user interface en het opsturen van HTTP requests naar de server. 
\subsubsection{Validation tests}
Data die wordt opgestuurd bij een HTTP request moet aan sommige eisen voldoen.Bij elke request zijn die eisen anders.
\\
\\
Voorbeeld : login request verwacht:
\\
Data = json dat username en password bevat
\begin{verbatim}
{
"username" : "skribl",
"password" : "Skribl123"
}
\end{verbatim}
Header = moet als content type application/json hebben.
\begin{verbatim}
{
"Content-Type" : "application/json"
}
\end{verbatim}

Validatie moet op dezelfde manier als op de server gebeuren. Het moet bij de twee tiers gebeuren voor security redenen.
\\
Indien dit niet zou gebeuren zou injection in de database mogelijk zijn!
\\
\\
%File : %pad naar de test

\subsubsection{HTTP requests test}
Probeert geldige HTTP requests te sturen naar de server en vergelijkt de verwachte response status met de verkregen status. 
\\
\\
%File : %pad naar de test

\section{Software risico gevallen}
De SKRIBL web-application bestaat uit drie tiers. Indien een van de tiers niet goed werkt (test falen) kunnen de andere tiers niet goed werken. Een perfecte werking van de drie tiers is nodig voor het opstarten van de web-applicatie.

\section{Functies}
De features die getest worden zijn:
\begin{itemize}
  \item Login 
  \item Logout 
  \item Register
  \item Delete user
  \item (Publicatie toevoegen) 
\end{itemize}

\section{Strategie}
Om onze code te testen gebruiken we Jasmine 2.0.3. Jasmine is een javascript test framework met heel veel kracht. Maar ook gebruiken we Postman, een chrome extension om HTTP requests te sturen.

\subsection{Jasmine 2.0.3}
Om code te testen in Jasmine moet een specification file worden aangemaakt. Deze specification file beschrijft de tests die dienen uitgevoerd te worden. 
\newline
Een specification file bestaat uit een of meerdere oproepen van de describe function. Describe komt overheen met een groep tests. Deze functie verwacht als argumenten een string en een functie die effectief de tests gaat implementeren.
\newline
De tests in deze describe functie worden aangemaakt aan de hand van de it functie. De it functie verwacht ook een naam-string en een functie, maar in deze functie worden de tests beschreven.
\newline
Om tests te beschrijven gebruiken we matchers. Matchers kunnen elementen vergelijken, controleren of het element wel gedefinieerd is en nog veel anderen! Het is ook mogelijk om zelf matchers te schrijven.
Voorbeelden van matchers :
\begin{itemize}
  \item expect().toBe();
  \item expect().toEqual();
  \item expect().toMatch();
  \item ...

\end{itemize}
Omdat we in Jasmine matchers kunnen schrijven wordt het mogelijk om alles in javascript te controleren.
\subsubsection{Tests uitvoeren}
Omdat Jasmine een standalone software is , is het uitvoeren van tests heel gemakkelijk. Wanneer een specification bestand geschreven is moet het pad naar deze alleen maar toegevoegd worden in een html bestand. Wanneer dit gedaan is dient het html bestand alleen maar geopend te worden in een webbrowser. In de webbrowser geeft Jasmine de geslaagde en gefaalde tests weer. Indien een test faalt wordt zijn describe naam en it naam getoond.

\subsubsection{Extra}
Een manual over Jasmine werd opgesteld door de test manager voor het SKRIBL-team. Deze staat in de referenties en legt het gebruik van Jasmine in detail uit.

\subsection{Postman}
Postman laat toe om op een heel simpele manier HTTP requests samen te stellen en vervolgens op te sturen aan de hand van een GUI. Het is daarna het werk van de tester om te kijken of de teruggegeven status de verwachte is of niet. In principe zijn alle HTTP responses tussen 200 en 299 juist.


\section{Stop criteria}
Wanneer een test item voor al zijn tests slaagt, kan de code weer aangepast worden en nieuwe tests geschreven worden. Indien dit niet het geval is , worden er geen nieuwe tests geschreven. Dit voorkomt tijdsverlies.

\section{Omgeving}
Jasmine heeft geen bepaald environment nodig om tests te kunnen uitvoeren . Het enige dat gedaan moet worden is de library uitbreiden met de gewenste source en specification bestanden.
\\
Daarentegen heeft Postman Google Chrome nodig, aangezien het een chrome extension is. 

\section{Verantwoordelijkheden}
In het SKRIBL-team schrijft iedereen zijn eigen tests. Indien een lid denkt dat zijn code af is, omdat het aan al zijn tests voldoet, bekijkt de Test Manager de tests en voegt er eventueel enkelen bij. De tests worden dan opnieuw uitgevoerd en het bestand eventueel opnieuw aangepast tot het voldoet aan alle tests. 

\section{Goedkeuring}
Wanneer de Test Manager alle tests heeft bekeken en geaccepteerd, en de code aan alle tests voldoet, moet de Quality Assurance Manager de code bekijken en deze wel of niet valideren.  Indien de Quality Assurance Manager de code niet valideert moet de programmeur zijn code aanpassen tot de code aan de eisen van de Test Manager en Quality Assurance Manager voldoet.
\\
Wanneer code wordt gevalideerd kan deze gepushed worden naar de master branch van de SKRIBL repository.


\clearpage

\end{document}
