\documentclass{article}
\usepackage[utf8]{inputenc}
\usepackage{natbib}
\usepackage{graphicx}
\usepackage{lastpage}
\usepackage{xcolor}
\usepackage{lipsum}
\usepackage[T1]{fontenc}
\usepackage{fancyhdr}
\usepackage{geometry}
\usepackage{url}
\usepackage[dutch]{babel}



\bibliographystyle{abbrv} %abbrvnat geeft problemen

\title{Software Test Plan}
\author{} %leave empty
\date{19 november 2014} %ok, manuele datum

\addtolength{\footskip}{1.3cm} % make more space for the footer
\pagestyle{fancyplain} % use fancy for all pages except chapter start
\lhead{}
\cfoot{\includegraphics[height=1.3cm]{Small_Logo.png}} % right logo
\rfoot{\thepage}
\renewcommand{\headrulewidth}{0.3pt} % remove rule below header
\renewcommand{\footrulewidth}{0.3pt} % remove rule below header

\begin{document}

\makeatletter
\begin{titlepage}

\newcommand{\HRule}{\rule{\linewidth}{0.7mm}} % Defines a new command for the horizontal lines, change thickness here


\vspace*{1.2mm}

\center 
\includegraphics[scale=0.6]{Logo.png}\\[1cm] 
%---------------------------------------------------------------------------------------
%	HEADINGS SECTION
%----------------------------------------------------------------------------------------

\textsc{\LARGE Vrije Universiteit Brussel}\\[0.3cm] % Name of your university/college
\textsc{\large WE-DINF-6537}\\
\textsc{\large Project Software Engineering}\\
\textsc{\large Academiejaar 2014-2015}\\[0.3cm] 
%\textsc{\large Software Engineering}\\[0.7cm] % Major heading such as course name

%----------------------------------------------------------------------------------------
%	TITLE SECTION
%----------------------------------------------------------------------------------------

\HRule \\[0.4cm]
{ \huge \bfseries \@title \\[0.5cm] }
\HRule \\[0.5cm]
 
%----------------------------------------------------------------------------------------
%	AUTHOR SECTION
%----------------------------------------------------------------------------------------

\Large
% laat voorlopig nog even de namen/mailadressen op deze pagina staan
% eerst moeten we zien of er een beter alternatief is om de lege plek dan op te vullen
% anders laten we ze gewoon staan...

%volgens mij ziet het er zo heel goed uit, maar als ze echt wegmoeten mss vub logo groter maken? 
% => OK, ik vind het ook  beter zo, we laten ze staan!
Douglas Horemans \textit{<dhoreman@vub.ac.be>}\\
Hannah Pinson \textit{<hpinson@vub.ac.be>}\\
Ivo Vervlimmeren \textit{<ivervlim@vub.ac.be>}\\
Noah Van Es \textit{<noahves@vub.ac.be>}\\
Pieter Steyaert \textit{<psteyaer@vub.ac.be>}\\

\vspace{0.6cm}

\includegraphics[scale=0.4]{VUB_schild.pdf}\\[0.5cm]

{\large 19 november 2014}
\vfill % Fill the rest of the page with whitespace

\end{titlepage}

\newpage
\section*{Versiegeschiedenis}
\addcontentsline{toc}{section}{Versiegeschiedenis}

\begin{center}
\begin{tabular}[t]{| c | c | c | c |}
    \hline
    \textbf{Versie} & \textbf{Datum} & \textbf{Auteurs\cite{note:author}} & \textbf{Beschrijving} \\
    \hline
    
    1.0     &  05/11/2014   &   \begin{tabular}{c} 
                                    Hannah Pinson \\
                                    Ivo Vervlimmeren \\
                                    Noah Van Es \\ 
                                    Pieter Steyaert \\
                                \end{tabular} & Aanmaak eerste versie \\
    \hline
    1.1     &  18/11/2014   &   \begin{tabular}{c} 
                                    Hannah Pinson \\
                                     \end{tabular} & \begin{tabular}{c}  
                                     			Gekregen feed-back  gedeeltelijk doorgevoerd\\
                                     			planning eerste sprint aangevuld \\
							 \end{tabular}  \\
    \hline

\end{tabular}
\end{center}
\newpage

\tableofcontents
\newpage


\section{Introductie}
Dit document is een test plan voor de SKRIBL web-application. SKRIBL is een webapplicatie die het mogelijk maakt voor onderzoekers om hun publicaties te beheren en kennis te maken met andere publicaties en mensen die hun interesse delen. Dit document beschrijft hoe de code getest wordt tijdens de eerste iteratie. Het doel van de eerste iteratie is dat de gebruiker kan inloggen en een publicatie kan toevoegen op alle gevraagde manieren. Voor de eerste sprint is enkel inloggen en uitloggen voorzien. 

\section{Referenties}
\begingroup
\renewcommand{\section}[2]{}% verwijdert titel sectie referenties
\bibliography{referenties}
\endgroup

%[I] Douglas als je een nieuwe versie schrijft kan je de referenties in de tekst zetten?
%--> \cite{naam1:naam2} in de tekst zelf en dan maak je een note in referenties.bib
%de refs worden dan automatisch opgelijst.
\begin{itemize}
  \item SDD versie 1.0
  \item Internal manual \url{http://wilma.vub.ac.be/~se4_1415/internalManuals/test-framework.pdf}
  \item Jasmine \url{http://jasmine.github.io/2.0/introduction.html}
  \item IEEE 829
\end{itemize}

\section{Definities}
\begin{description}
\item[HTML]Hyper Text Markup Language. Is een taal dat de inhoud van een webpagina beschrijft.
\item[DOM]Document Object Model. Is een soort boom dat de content van een webpagine inhourd.
\item[GUI]Graphical User Interface.
\item[API]Application Programming Interface.
\item[MVC]Model view controller. Is de link tussen de front-end van een web app en de server(in een Model view).
\end{description}

\section{Test items}

De items die gedurende de eerste iteratie getest worden zullen zeker nog evolueren. Hierdoor zullen de tests  herschreven of aangepast moeten worden voor elke iteratie en heeft elke test een versienummer.

\begin{table}[h]
\begin{tabular}{l | c }
 Item&Test versie\\
 \hline
 Persistens test&1.0\\
 Validation test&1.0\\
 Vieuw test&1.0\\
\end{tabular}
\end{table}

\subsection{Persistentie test}
Deze test gaat controleren of alle gegevens "persistent" zijn opgeslagen in de database. 
\newline
Deze test zal hiervoor gebruikmaken van de database API. Na gegevens te hebben toegevoegd zal de test diezelfde gegevens opvragen om de persistentie ervan te controleren. Ook zal de test gegevens proberen te modificeren en deze daarna opvragen , om te controleren of veranderingen van gegevens zonder fout worden doorgevoerd. 

\subsection{Validatie test}
In de GUI worden sommige ingevoerde veldjes gevalideerd. Deze test moet uitmaken of de validatie correct gebeurt. Elke validatietest heeft andere validatie regels en vereist hierdoor een eigen specifieke test.

%wat gaan de DOM tests testen?
\subsection{View test}
View test gaan de MVC controleren. De test controleert of de juiste controllers opgeroepen worden met de juiste argumenten. Verder zal dit ook tests uitvoeren op het DOM. 

\section{Software risico gevallen}
De belangrijkste tests van de eerste iteratie zijn de persistentie tests. Deze controleren dat de database goed werkt. Indien deze niet goed gestructureerd, of ge{\"i}mplementeerd zijn werkt geen enkel onderdeel van het programma!
\newline
Een ander risico wordt veroorzaakt door het feit dat het SKRIBL-team agile werkt. Agile werken kan veranderingen in de API's veroorzaken. Dit kan de tests doen falen, of de uitkomsten zeer beinvloeden.

\section{Functies}
De functies die getest moeten zijn.
\begin{itemize}
  \item Login 
  \item Logout 
  \item Register
  \item Publicatie toevoegen 
\end{itemize}

\section{Strategie}
Om onze code te testen gebruiken we Jasmine 2.0.3. Jasmine is een krachtig javascript test framework.
\newline

Zoals eerder vermeld is het testen van de GUI moeilijker dan het testen van pure javascript code. Maar dit is geen groot probleem vermits javascript aan het DOM kan en via het DOM kan Jasmine aan alle html onderdelen. 

\subsection{Jasmine 2.0.3}
Om code te testen in Jasmine moet een specification file worden aangemaakt. Deze specification file beschrijft de tests die dienen uitgevoerd te worden. 
\newline
Een specification file bestaat uit een of meerdere oproepen van de 'describe' functie. Describe komt overeen met een groep tests. Deze functie verwacht als argumenten een string en een functie die daadwerkelijk de tests gaat implementeren.
\newline
De tests in deze describe worden aangemaakt aan de hand van de 'it' functie. De 'it' functie verwacht ook een naam-string en een functie. Maar in deze functie worden de tests beschreven.
\newline
Om tests te beschrijven gebruiken we matchers. Matchers worden gebruikt om elementen te vergelijken, te controleren of het element wel gedefinieerd, enz. Het is ook mogelijk om eigen matchers te defini{\"e}ren.
\newline
Omdat we in Jasmine matchers kunnen schrijven wordt het mogelijk om alles in javascript te controleren.

\subsection{Tests uitvoeren}
Omdat Jasmine standalone software is, is het uitvoeren van tests heel makkelijk. Wanneer een specification file geschreven is moet het pad naar deze alleen maar toegevoegd worden in een html file. Wanneer dit gedaan is dient de html file geopend te worden in een webbrowser. In de webbrowser maakt Jasmine een visualisatie van de geslaagde en gefaalde tests. Indien een test faalt wordt zijn 'describe' naam en 'it' naam weer gegeven.

\subsection{Test criteria}
In Jasmine is een test geslaagd of niet. Een test kan niet half geslaagd zijn.

\subsection{Extra}
Een manual over Jasmine werd opgesteld door de test manager voor het SKRIBL-team. Deze staat in de referenties en legt het gebruik van Jasmine in detail uit. 

\section{Stop criteria}
Zolang een file niet voor al zijn huidige tests slaagt worden er geen nieuwe tests geschreven. Wanneer dit wel het geval is worden er nieuwe tests geschreven en vervolgens nieuwe features geimplementeerd. Dit voorkomt dat er onnodige tests geschreven worden, en tijd word verspild, voor een file dat nog niet aan de initi{\"e}le tests voldoet.

\section{Omgeving}
Jasmine heeft geen specifieke omgeving nodig om tests te kunnen uitvoeren . Het enige dat gedaan moet worden is de library uitbreiden met de gewenste source files en specification files.

\section{Verantwoordelijkheden}
In het SKRIBL-team schrijft iedereen zijn eigen tests. Indien een member denkt dat zijn code af is (omdat het aan al zijn tests voldoet) bekijkt de test manager de tests en voegt er eventueel bij. De tests worden dan opnieuw uitgevoerd en de file eventueel opnieuw aangepast tot het voldoet aan alle tests. 

\section{Goedkeuring}
Wanneer de Test Manager alle tests heeft bekeken en geaccepteerd, en de code aan alle tests voldoen, moet de Quality Assurance Manager de code bekijken en deze wel of niet valideren. 


\clearpage

\end{document}
